# This file will not compile, its just a holding spot for things im not sure what to do with....


\emph{We consider a new generation of stochastic volatility models, dubbed by Jim Gatheral, Thibault Jaisson and Mathieu Rosenbaum as `rough volatility models’, where the instantaneous volatility is driven by a (rough) fractional Brownian motion} %https://sites.google.com/site/roughvol/


\subsection{Types of volatility models}
Models for estimating volatility can be divided in two types, historic and implied volatility estimation.

\textbf{Implied volatility models:}

Also known as forward-looking models. Their principal output are the option prices observed in the market. Therefore, it can be thought as the volatility value needed to make a model pricing, such as Black-Scholes-Merton, to have the same option price as the price observed in the market. Plotting implied volatility as a function of strike price and time to expiry generates the volatility surface. If we would like to observe the dynamics of the volatility, we could use the Dupire’s local volatility model, where the local volatility $\sigma(Y_{t}, t)$ is a deterministic function of the underlying price and time.  

\textbf{Historical estimation:} 

Also known as models of backward-looking, where the principal input for its estimation is the historical data of the asset prices. Among this type of estimation we can also divide the models as: i) Punctual measures and ii) Series measures. The latter can also be divided in parametric models and non parametric models.
 
Punctual measure is the simplest estimation as it can only be the standard deviation of the returns of a given asset. However, the main disadvantage is that the evolution of this parameter is not taken into consideration, as it will remain constant for different periods of time. 

Therefore it is immediate to assume that considering the evolution of a given time period will generate a better accuracy of this parameter. For example, the models of autoregressive conditional heteroscedasticity such as ARCH and GARCH models assume the volatility to be a function of the return of the prices but also of the volatility itself for previous periods of time. These type of models are widely used and have been developed for example to ensure an asymmetric volatility on positive and negative returns (e.g EGARCH model).
%https://file.scirp.org/pdf/JMF_2017051916361813.pdf

Non parametric models have the great advantage of not assuming a given distribution of the data. However, a disadvantage is that a great amount of data to fit correctly the model, which in real life it can be hard to access for. Examples for this estimation can be neuronal networks, kernel regressions, etc.

Finally, parametric models assume a distribution of our data. For example, Brownian motion assume a normal distribution for the return of the prices. Stochastic models are part of this type, and even though it was not quite often used because of its complexity, now we can use computational power to .....


Stochastic models:

Black-Scholes framework, the volatility function is either constant or a deterministic function of time.  Notable amongst such stochastic volatility models are the Hull and White model [32], the
Heston model [31], and the SABR model [29]. Whilst stochastic volatility
dynamics are more realistic than local volatility dynamics, generated option
prices are not consistent with observed European option prices

Fact that "smile dynamics" is poorly predicted by local vol models leading to bad Hedging of exotic options.

More recent market practice is to use local-stochastic-volatility
(LSV) models which both fit the market exactly and generate reasonable
dynamics.